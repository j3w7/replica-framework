\chapter{Conclusions}

%\section{Results}

The Replica Framework provides  means for implementing Collaborative
Ontology Development (COD) tools and Distributed Ontology Systems (DOS).
It  allows synchronous working on an ontology with
multiple users. 
It is first of its kind that combines approaches from COD and DOS in a unified framework.
The ECF-based architecture provides a reliable basis for distributed communication.
The framework is flexible to realize a range of COD, DOS and hybrid scenarios. The modular
structure of the framework has been created with extensibility in mind
and the shared ontology builder allows migration to a new OWLAPI version
to reduce maintenance costs when keeping the framework up-to-date
which also happened in the course of the development. In addition, every component can be configured individually.

Unit tests for all major functionalities of each module of the code
base are provided as a means to foster rapid development of stable applications and framework extensions.

Initial experience shows that the speed of change processing was sufficient in unit tests and the
demonstrator but there are various possibilities to improve it.
For example, collecting changes and applying them all at once was 
a lot faster than applying each change on its own.


Many features remain to be implemented in future work. For example
\emph{query management} has been omitted as well as \emph{policy management}.
While the current implementation of the Replica Framework supports distribution
of OWL ontologies, import dependencies between ontologies are not yet supported.


Another aspect that has not been addressed is how different shared ontology
fragments are synchronized when they are connected to a logical
shared ontology. This could be done for example by computing a logical
diff of the ontologies such as CEX and MEX   \cite{konev2008} (implemented in OWLDiff\footnote{\url{http://krizik.felk.cvut.cz/km/owldiff/}}) using the result to gain synchronicity.

Equally important is the question of fault tolerance. While the
Replica Framework implementation relies on transactional message communication
error correction has not been addressed heavily and for example
collecting changes locally and re-applying them later on when the connection
is lost has not been implemented yet.

The Replica Framework implementation offers the possibility to configure
every component from a central place. This is currently done programmatically
and will be extended to file-based configuration in the future.

Concerning the Replica Framework implementation not all of the aspects of
COD have been addressed. For example,
private/shared workspace support could be integrated together with
asynchronous ontology access. The framework model forms a platform for implementation of many of these aspects. A search function and version control can be implemented
by using the \emph{change management} components that can also be used
to incorporate workflow support and by functions of
the communication module. Implementing tools that assist in the
communication is easy by using the communication module or the underlying
ECF methods directly.

%While the emphasis has been placed on the Collaborative Ontology Development
%aspect does the Replica Framework model also take account of the Distributed
%Ontology System aspect.

The NeOn plugin and demonstrator presented in chapter \ref{chp-demonstrations} is a proof of
concept and has shown that the Replica Framework is functional and stable.
In the future, we plan further tests of  the demonstrator with use case ontologies.

%\section{Discussion and Outlook}


%Some aspects of Collaborative Ontology Development could not have been
%implemented yet but were kept in mind during development. Extending the
%Replica Framework implementation should therefore not be hard.


