\chapter{Data sheets}\index{Data sheets}


\textcolor{darkred}{Anmerkung: Wenn das Thema der Diplomarbeit auch Hardware-Komponenten bzw. den Aufbau eines Demonstrators eingeschlossen hat, so ist ein Anhang mit den wichtigsten Datenblättern sehr sinnvoll. Zum Einen können interessierte Leser direkt ohne Internetrecherche die Betriebsparameter der Komponenten einsehen, zum Anderen ist somit auch eine gute Dokumentation des Systems für die Bedienung durch andere Anwender als den Autor gegeben. Auch wenn die Datenblätter normalerweise online verfügbar sind, so erspart der beigefügter Anhang dem Anwender eine aufwändige Recherche. Die Anzahl Seiten sollte 25--30 nicht überschreiten.}

\textcolor{darkred}{Genau wie bei den Quelltextabschnitten im Anhang muss aber auch bei den Datenblättern ein kurzer Abschnitt vorweg geschickt werden, welcher die Auswahl der Datenblätter und die Relevanz erklärt.}

\textcolor{darkred}{Die einzufügenden Datenblätter sollten im pdf-Dateiformat vorliegen und nicht als Grafik, sondern als ganze Seite einzufügen.}


Die nachfolgenden Datenblätter erläutern Systemparameter, Funktionsweise, Anschlussvarianten und Betriebsarten zu dem im Rahmen der vorliegenden Arbeit verwendeten Maxon-Motorregler ADS\_E 50-10.



\includepdf[
   pages={-},
   nup=1x1,
   landscape=false,
   noautoscale=false,
   turn=false,
   scale=0.8,
   trim= 0mm 0mm 0mm 0mm,
   clip=true,
   pagecommand={},
   delta=0mm 0mm
]{BilderAnhangD/Seiten_aus_ads_e50_10_de.pdf}

%
% Vorsicht: Pfad- und Dateiname darf keine Leerzeichen enthalten
%

