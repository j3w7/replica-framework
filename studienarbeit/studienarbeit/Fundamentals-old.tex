\chapter{Fundamentals}

\section{Semantic Web}
The problem with information on the internet today is, that it is designed
to be used by humans. Consider you want to find all Wikipedia articles
about musical artists who were born in Germany and lived before 1900.
While Wikipedia covers this information
implicitly in its articles you can not easily express a query for this and
would have to read all articles to find those you are looking for.\\
The semantic web approach develops languages for expressing such knowledge
in a machine processable form.
By adding semantic data to articles it would be possible for a machine
to recognize and deal with the content. The path to the vision of the
Semantic Web is about adding machine processable meta information or
metadata to the existing web not creating a seperate new web \cite{berners-lee98}.


\subsection{Ontologies}
\subsubsection{Overview of ontology languages}
\index{DAML+OIL}
\index{F-Logic}
\index{Topic Maps}
\index{KIF}
An ontology is a "formal, explicit specification of a shared conceptualisation"
\cite{gruber1993}. It is used to represent knowledge in a machine readable form.
\todo{explain}\\
There are in fact quite a few ontology languages:
RDF-S, DAML+OIL\footnote{The DARPA Agent Markup Language Homepage, \url{http://www.daml.org/}}
which is the predecessor to OWL, F-Logic\footnote{Frame Logic, \url{http://forum.projects.semwebcentral.org/forum-syntax.html}}, Topic Maps and KIF.\\
This section will only deal with the languages most relevant to this
seminar paper.


\subsubsection{XML, RDF and RDF Schema}
\index{RDF}
\index{RDF Schema}
\todo{XML dialect}
\index{XML}
The \emph{eXtensible Markup Language} (XML) is the key framework for interchange
of data and meta information between applications. But XML does not define
any semantics only syntax.\\
The \emph{Resource Description Framework} (RDF) is a \emph{data-model}
developed in 2004.
It has clearly defined semantics. The basic principle of RDF is to
express knowledge in \emph{statements}. A statement is a triple consisting
of a subject a predicate and an object.
It provides the means to express sub class relationships and .

\subsubsection{OWL}
\index{OWL}
\todo{Other Langs}
The most popular language to express knowledge in form of an ontology is
the \emph{Web Ontology Language} (OWL) developed by the W3C.
OWL makes up a fundamental part of the semantic web.
OWL is designed not only to to formulate but also to exchange and reason
with knowledge about a domain of interest.\\
OWL poses three levels of expressiveness OWL Lite, OWL DL and OWL Full.
OWL Lite was meant to be implemented easily and to get users started quickly
with the language.\\
OWL DL (where DL stands for "Description Logic") was designed to support
the existing Description Logic business segment and to provide computational
properties for reasoning systems.\\
For more see the OWL 2 Web Ontology Language Primer \cite{OWL2Primer}.\\
\todo[inline]{Example, description of axiomatic structure}
\vspace{1cm}
In the following example a class called \emph{CheeseTopping} is defined
along with its portuguese label and the subclass relation with a class
called \emph{PizzaTopping}.
If the class PizzaTopping has not yet been defined anywhere else it is
implicitly asserted to exist.
\begin{tabbing}
\hspace*{1cm}\=\hspace{1cm}\=\hspace{1cm}\=\kill
\verb$ <owl:Class rdf:about="#CheeseTopping">$\\
\>\verb$ <rdfs:label xml:lang="pt">CoberturaDeQueijo</rdfs:label>$ \\ 
\>\verb$	<rdfs:subClassOf>$\\
\> \> \verb$		 <owl:Class rdf:about="#PizzaTopping"/>$\\
\>\verb$	</rdfs:subClassOf>$\\
\verb$ </owl:Class>$\\
\end{tabbing}

\subsubsection{Ontologies vs Databases}
There is an obvious analogy to databases but there are important differences
between ontologies and databases.
In contrast to databases do ontologies have an \emph{open world assumption}:
The truth-value of a statement unknown to an observer is assumed to be true.
Information is assumed to be incomplete by default. Consider a query to
a database for a certain telephone number. If there is no entry for the
name you are looking for in the database the query will evaluate to false
which would denote there is no such number. In comparison to that when
asking an ontology you could not tell whether there is a number or not
because it is just unknown.\\
Moreover unlike databases ontologies do not assume that instances have
a unique name. An instance can be refered to with more than one name.
When talking about a cape we can also refer to it by saying cloak but we
mean the same thing. In a database these would be two different entities
whereas in an ontology it can be expressed that both names refer to the
same concept.\\
Finally whilst database schemas behave as constraints on the structure of
data defining legal database states, ontology axioms behave like logical
implications and entail implicit information.\todo[inline]{example?}
for more refer to \cite{horrocks2008}


\subsection{Reasoning}

\subsection{Distributed Ontology Systems}
For reasoning on very large ontologies the performance of current
reasoners is clearly not sufficient \cite{chen09}. An initial approach to
cope with this limitation is to distribute the query processing across a
set of nodes.


\subsubsection{Components of a DOS}
As proposed by \todo{cite: framework} a \emph{Distributed Ontology System} (DOS)
consists of three main components:
\begin{enumerate}
\item A large ontology
\item A Set of ontology fragments
\item A Dictionary $D$
\end{enumerate}

\subsubsection{Ontology modularisation}
The modularization of an ontology is a very important principle when
it comes to Distributed Ontology Systems.
The goals of ontology modularization include but are not limited to reuse,
scalability for information retrieval and reasoning as well as for evolution and
maintenance, complexity management, understandability and personalization.
Let $\mathcal{O}$ be an ontology over an alphabet $A_k$ and $\mathcal{F}$
be a \emph{fragment} over $A_k$. $\mathcal{F}$ is a subset of statements
of $\mathcal{O}$.
Ontology fragments may contain complete or incomplete A-Boxes, T-Boxes
or R-Boxes.
$\mathcal{F}$ is called a \emph{module} if a query $q$ performed against
it returns the same result as a query performed against $\mathcal{O}$.
Therefore it is useful to differentiate between mondular and non-modular
fragments which is basically the question whether the fragments are disjoint
or overlap.\\
Quering non-modular ontology fragments $\mathcal{F}_1,\mathcal{F}_2$ of
an ontology $\mathcal{O}$ poses several problems:
(1) There might be intrinsic knowledge of $\mathcal{O}$ which cannot be
infered of either $\mathcal{F}_1$ or $\mathcal{F}_2$ alone.
(2) \todo[inline]{more problems}.




\section{Object replication}




\section{Definitions}
asds

\begin{description}
  \item[First] \hfill \\
  The first item
  \item[Second] \hfill \\
  The second item
  \item[Third] \hfill \\
  The third etc \ldots
\end{description}










%






