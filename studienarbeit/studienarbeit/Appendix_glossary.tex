\chapter{Glossary}\index{Glossary}
\label{anhang_e}



%\textcolor{darkred}{Anmerkung: Das vorliegende Glossar wurde ohne die Zuhilfenahme der speziellen Glossarumgebungen von Latex erstellt, um eine etwas freiere Formatierung nutzen zu können.}
%\medskip



\interlinepenalty=10000 % keine Schusterjungen, keine Hurenkinder

\begin{description}

\item[\bf{AOP}] Aspect Oriented Programming.
\item[\bf{COD}] Collaborative Ontology Development.
\item[\bf{DAML}] Darpa Agent Markup Language.
\item[\bf{DOS}] Distributed Ontology System.
\item[\bf{ECF}] Eclipse Communication Framework.
\item[\bf{EPL}] Eclipse Public License.
\item[\bf{GUI}] Graphical User Interface.
\item[\bf{IDE}] Integrated Development Environment.
\item[\bf{OOP}] Object Oriented Programming.
\item[\bf{OWL}] Web Ontology Language.
\item[\bf{RDF}] Resource Description Framework.
\item[\bf{RDFS}] Resource Description Framework Schema.
\item[\bf{URI}] Uniform Resource Identifier.
\item[\bf{XML}] eXtensible Markup Language.


\end{description}

\interlinepenalty=100



