\chapter{Introduction}
\label{chapter_einfuehrung}



\section{Motivation, objectives and contributions}
\label{section_motivation}
There is a need for collaborative ontology development tools.
In fact, first contributions to the field of collaborative
ontology development were made in the outcomes of the
\emph{Knowledge Sharing Effort} initiated by the Defense Advanced
Research Projects Agency (DARPA) back in 1990 with the \index{Ontolingua}\emph{Ontolingua Server}.
Nowadays the development of large complex ontologies such as the Biomedical Grid 
Terminology (\index{BiomedGT}BiomedGT)\footnote{\url{http://biomedgt.nci.nih.gov/}}
involve many scientists from all around the globe and requires collaborative
tools to assist in the process of creation which is organized in workgroups
with members from many different countries.

In the vision of the semantic web \cite{berners-lee01} there will not be a single large ontology
but rather many ontologies scattered throughout the web with content
specific to their creators domain but interlinked as to the principles of
Linked Data.
Knowledge will be fragmented in smaller pieces of information rather than
residing in one global knowledge base accessible from a single host.
Therefore distributed ontology systems are required.

%The amount of ontologies is becoming very large and our knowledge of
%reasoning over web scale ontologies is still at a very early stage.
%There is still a large discrepancy between the required performance and
%the performance that current reasoning systems expose.
%The Large Knowledge Collider (LarKC)\footnote{\url{http://www.larkc.eu/}}
%is a pioneer project in this field which tries to remove the currently 
%existing scalabilty barries by applying a distributed reasoning approach.

There have been several successful approaches for Collaborative Ontology
Development and Distributed Ontology Systems, but they have been dealt with rather separately.
The concept of a distributed ontology is in fact quite similar to the
situation in Collaborative Ontology Development where many different authors
develop specific parts of an ontology: A large ontology is divided into
smaller parts either physically or logically which together form the ontology.

This seminar paper presents a novel framework designed to met requirements
of both fields as an initial step to improve the process of building 
semantic web applications with a need for large scale distributed
ontologies and collaboration tools.




%\section{Structure of this document}
%This work starts with an introduction to the Semantic Web mediating
%fundamentals. Followed by the description of the Replica Framework architecture.
%The architecture section is the main subject of this work.
%After this section follow descriptions of the implementations of the
%backend and demonstrations. In the last chapter results are discussed
%and an outlook of further development is made.

%2 Semantic Web Fundamentals 
%3 Framework architecture 
%4 Implementation of the Backend
%5 Implementation of Demonstrations
%6 Conclusions 
%A Source code 
%B Glossary 
%Continuative literature of scientific work 


%\verb*$sdcc -I c:\sdcc\include -L c:\sdcc\lib\large simpletest.c --model-large$


\clearpage
